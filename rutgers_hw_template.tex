\documentclass{rutgers_hw}
\usepackage{rutgers}
% \usepackage[none]{hyphenat} % Use to avoid hyphens

\author{Firstname Lastname} % Enter your name
\netid{abc123} % Enter your NetID or comment out
\collaborators{Leonhard Euler, Bernard Bolzano} % Enter your collaborators or comment out
\assignment{Assignment 1} % Enter the assignment name
\date{\today} % Replace with due date
\course{Course Name} % Enter the course name
\semester{Fall 2021} % Enter the semester
\sectionnum{123} % Enter your section number
\instructor{Professor Name} % Enter your professor's name
\institution{Rutgers University} % Enter your university

\begin{document}

\maketitle

% You can comment out code by using the percentage symbol


\begin{exern}{1.2.3} % This is a manually numbered exercise.
  Exercise text
\end{exern}
\begin{solution} % This is a solution environment.
  Your answer here.
\end{solution}

\begin{exer} % This is an automatically numbered exercise.
  Exercise text
\end{exer}
\begin{proof} % This is a proof environment.
  Your proof here. \par

  You may want to write an induction proof. You can use the induction commands as follows:
  \begin{induction}
    \begin{basecase}
      This is your base case.
    \end{basecase}
    \begin{indhyp}
      This is your induction hypothesis.
    \end{indhyp}
    \begin{indstep}
      This is your induction step.
    \end{indstep}
    Thus by the principle of mathematical induction, we have proven our theorem. \qedhere % The command \qedhere places the QED symbol on the same line, instead of adding another line.
  \end{induction}
\end{proof}

\begin{exeru} % This is an exercise that is not numbered.
  Exercise text
\end{exeru}

\begin{thm} % This is an automatically numbered theorem. For a theorem that isn't numbered, use thmu instead; for a manually numbered theorem, use thmn and add number in a set of curly braces.
  This is a theorem.
\end{thm}

\begin{lem} % This is an automatically numbered lemma. For a lemma that isn't numbered, use lemu instead; for a manually numbered lemma, use lemn and add number in a set of curly braces.
  This is a lemma.
\end{lem}

\begin{defn} % This is an automatically numbered definition. For a definition that isn't numbered, use defnu instead; for a manually numbered definition, use defnn and add number in a set of curly braces.
  This is a definition.
\end{defn}



\end{document}
